%-
%- DONE

\chapter{Pyomo Modeling Strategies}
\label{chap:overview}

\abstract{This chapter illustrates different strategies for formulating 
and optimizing algebraic optimization models using Pyomo. We provide a 
brief overview of the core modeling components supported by Pyomo.  Then, 
we describe how to formulate both concrete and abstract models with Pyomo.  
Finally, we provide a brief tutorial on how these models can be analyzed with the \code{pyomo} command.}

\section{Modeling Components}

Pyomo supports an object-oriented design for the definition of
optimization models.  The basic steps of a simple modeling process are:
\begin{enumerate}

\item Create model and declare components 
\item Instantiate the model
\item Apply solver
\item Interrogate solver results

\end{enumerate}
In practice, these steps may be applied repeatedly with different data or with 
different constraints applied to the model.  However, we focus on this
simple modeling process to illustrate different strategies for modeling with
Pyomo.

A Pyomo {\em model} consists of a collection of
modeling {\em components} that define different aspects of the model.\index{model}\index{modeling component}
Pyomo includes the modeling components that are commonly
supported by modern AMLs:  index sets, symbolic parameters, decision
variables, objectives, and constraints.
These modeling components are defined in Pyomo through the following Python classes:
\begin{center}
\begin{tabular}{ll}
\code{Set} & set data that is used to define a model instance\index{Set component@\code{Set} component}\index{set!Set@\code{Set} component}\index{data!set} \\
\code{Param} & parameter data that is used to define a model instance\index{Param component@\code{Param} component}\index{parameter!Param@\code{Param} component}\index{data!parameter} \\
\code{Var} & decision variables in a model\index{Var component@\code{Var} component}\index{variable!Var@\code{Var} component} \\
\code{Objective} & expressions that are minimized or maximized in a model\index{Objective component@\code{Objective} component}\index{objective!Objective@\code{Objective} component} \\
\code{Constraint} \hspace{0.2in} & constraint expressions in a model\index{Constraint component@\code{Constraint} component}\index{constraint!Constraint@\code{Constraint} component}
\end{tabular}
\end{center}
Two model classes provide a context for defining and initializing 
these modeling components: \code{ConcreteModel} and \code{AbstractModel}.\footnote{In versions
1.x and 2.x of Pyomo, the \code{Model} class serves as an alias for the \code{AbstractModel}
class. Both the aliasing and the \code{Model} class will be deprecated in Pyomo 3.1 and 
subsequent releases.} \index{ConcreteModel component@\code{ConcreteModel} component}\index{model!ConcreteModel@\code{ConcreteModel} component}\index{AbstractModel component@\code{AbstractModel} component}\index{model!AbstractModel@\code{AbstractModel} component}
For example, modeling components can be directly added to an \code{AbstractModel} object as an
attribute of the object:
\listing{examples/overview/simple1.py}{body}{4}{6}
The \code{model} object is a class instance of the \code{AbstractModel} class, and \code{model.I} is a 
\code{Set} object that is contained by this model.\index{class instance}
Many modeling components in Pyomo can be optionally specified as {\it indexed components}:
collections of components that are referenced using one or more values\index{indexed component}.
In this example, the parameter \code{p} is indexed
with set \code{I}. 

\begin{Xnotebox}
For simplicity, many short examples in this book omit the Python import statement for \code{pyomo.core}\index{pyomo.core package@\code{pyomo.core} package}.  These examples assume that this package has been previously imported with the following command:
\begin{quote}
\begin{lstlisting}[frame=none]
from pyomo.core import *
\end{lstlisting}
\end{quote}
This command imports all of the classes, functions and data from the \code{pyomo.core} package;  see Appendix~\ref{chap:python} for further details about 
importing Python packages.
\end{Xnotebox}

The following sections describe different strategies for creating
models with Pyomo.  We begin with strategies for generating concrete
models and progress to more general strategies for abstract models.
To illustrate this flexibility, we consider a generalization of the 
simple LP that we introduced in Chapter~\ref{chap:intro}:
\begin{equation}
\label{eqn:simple-abstract}
\begin{array}{lll}
\min & \sum_{i=1}^n c_i x_i &\\
\st & \sum_{i=1}^n a_{ij} x_i \geq b_j \hspace{0.25in} & \forall j = 1 \ldots m\\
    & x_i \geq 0 & \forall i = 1 \ldots n
\end{array}
\end{equation}
The following LP is a model instance (i.e., a model object with fully specified data)\index{model instance} of the abstract LP ~\ref{eqn:simple-abstract}:
\begin{equation}
\label{eqn:simple-concrete}
\begin{array}{ll}
\min & x_1 + 2 x_2\\
\st & 3 x_1 + 4 x_2 \geq 1\\
    & 2 x_1 + 5 x_2 \geq 2\\
    & x_1, x_2 \geq 0
\end{array}
\end{equation}
The examples in this chapter illustrate different ways of formulating both
abstract and concrete models.

We conclude this chapter with a brief introduction to the \code{pyomo}
command, which can be used to optimize Pyomo models within a Unix or
Windows command shell.\index{pyomo command@\code{pyomo} command}  Additionally, we summarize how Pyomo model
scripts can be setup to construct and solve model instances\index{model!instance}.  Note that
additional detail about Pyomo modeling components can be found in
Chapter~\ref{chap:components2}, Chapter~\ref{chap:components1}, and Chapter~\ref{chap:components3}.  Additional detail
about the \code{pyomo} command and scripting with Pyomo models can be
found in Chapter~\ref{chap:command}.


\section{Concrete Models: Specifying Components Via Expressions}

\index{concrete model}
The simplest way to express a concrete model is by directly creating
the modeling components with expressions that specify numerical values;
the alternative, two-step process involving an abstract model is described
in Section~\ref{chap:overview-abstractmodels}.
The \code{ConcreteModel} class is used to define these types of models:
\listing{examples/overview/concrete1-parts.py}{model}{4}{4}
In contrast to \code{AbstractModel} objects, 
model components contained in\linebreak \code{ConcreteModel} objects
are immediately initialized as they are added to
the model instance.\index{model!initialization}

Decision variables are required to define non-trivial expressions in Pyomo. In the
simplest case, we can define each decision variable separately:
\listing{examples/overview/concrete1-parts.py}{var}{8}{9}
The \code{within} argument used in these declarations constrains the values of
the defined variables to the set of non-negative real numbers, in this case by
having Pyomo to automatically generate the associated lower bound constraints\index{variable!bounds}.

An optimization objective is defined by passing an algebraic expression involving
numeric constants and Pyomo variables as an argument -- via the \code{expr} keyword --
to the \code{Objective} class constructor\index{objective!expression}:
\listing{examples/overview/concrete1-parts.py}{obj}{13}{13}
The objective expression is defined implicitly by Pyomo, using overloading of the Python
multiplication and addition operators. This is done automatically so the modeller need only
give the expression as shown in the example.  Similarly, constraints of a concrete model are
defined by passing an expression as an argument -- again via the \code{expr} keyword -- 
to the \code{Constraint} class constructor\index{constraint!expression}:
\listing{examples/overview/concrete1-parts.py}{con}{17}{18}
Note that this expression additionally specifies a constraint bound, via overloading
of the Python \code{>=} operator.
Example~\ref{ex:concrete1} includes the entire formulation for this concrete model\index{constraint!bounds}.

For large concrete models, the use of explicit variables will be cumbersome, since this requires separate 
declarations for each variable in the model.  To address this issue, Pyomo supports the notion of variable 
\emph{indexing}. For example, the variable \code{x} can be defined with an explicit index set\index{indexed component}:
\listing{examples/overview/concrete2-parts.py}{var}{6}{6}
This indexed variable can be used to specify the objectives and constraints with a natural, extensible syntax:
\listing{examples/overview/concrete2-parts.py}{expr}{10}{12}
Indexed variables allow logically related sets of variables to be grouped under a common name, and naturally
differentiated via a sub-script.

Example~\ref{ex:concrete2} includes the formulation for the indexed concrete model of Formulation(~\ref{eqn:simple-concrete}).

A limitation of the formulation in Example~\ref{ex:concrete2}
is that the data used is explicitly expressed in the specification of
the index set for \code{x} and the numerical values in the constraints
and objective. Any change in data values requires potentially extensive
modifications to the concrete model. Alternatively, this data can be
symbolically specified using standard Python data structures and subsequently
used in the formulation of the expressions for objectives and constraints.  
For example, the following declarations define Python dictionary objects 
that represent the data used in the objectives and constraints for
Formulation~(\ref{eqn:simple-concrete}):
\listing{examples/overview/concrete3a-parts.py}{Xdata}{4}{7}
Using this symbolic data, the objective can be simply redefined as follows:
\listing{examples/overview/concrete3a-parts.py}{obj}{13}{13}
An even more extensible formulation of this objective uses Python 
iteration syntax to sum over a set of related (indexed) decision variables.  
The Python \code{sum} function and {\em generator} syntax can be used 
to provide a concise specification of a summation\index{python!generator syntax}\index{python!sum@\code{sum} function}:
\listing{examples/overview/concrete3-parts.py}{obj}{13}{13}
This syntax specifies that the terms \code{c[i]*model.x[i]} are generated
by iterating over the set \code{N}.  As these terms are generated, the
function \code{sum} adds them together to form the overall expression.
This type of inlining of summations is a powerful feature that provides
a syntax that is similar to that used in AMLs with custom languages.
Example~\ref{ex:concrete3} includes the complete formulation of the concrete model using these
external data declarations.


\section{Concrete Models: Specifying Components Via Rules}

Although the abstract formulation in Formulation~(\ref{eqn:simple-abstract})
specifies a set of constraints, each constraint declared in
Example~\ref{ex:concrete3} is defined separately.  As in the case of 
decision variables, this modeling approach will become cumbersome
for large models.

Pyomo addresses this and other, related issues by allowing modeling components
to be initialized with user-defined functions, which we call 
{\em rules}.\index{constraint!rule}  The idea is that complex initialization of a
collection of constraints or objectives (for example) can be managed by a function 
that generates each constraint or objective expression individually\index{objective!expression}\index{constraint!expression}.  Similarly, rules
can be used to construct complex sets and parameters in a generic manner.

Example~\ref{ex:concrete4} illustrates the use of a rule to define the
constraints for Formulation~(\ref{eqn:simple-concrete}) in a generic manner.
The index set \code{M} defines the indices of the constraint \code{con}, and
the function \code{con\_rule} is used to construct the individual constraint
expressions\index{indexed component}. The arguments to this rule are the constraint indices\index{constraint!index}, in addition 
to the model instance that is being constructed. As with all components added
to a \code{ConcreteModel} object, initialization is immediately executed, for
all values of the specified index set. Note that the name of a rule 
used to define a constraint (or other modeling component) is not restricted.
However, some advanced Pyomo features require the use of unique rule names
(e.g. see Chapter~\ref{chap:scripts}).

\index{component!rule}
Although the function arguments for component rules are similar for all
component types, the expected type of the values returned are different:
\begin{center}
\begin{tabular}{ll}
\code{Set} & rule must return a Python set or list object.\\
\code{Param} & rule must return an integer or float value.\\
\code{Objective} & rule must return an expression\index{objective!expression}. \\
\code{Constraint} \hspace{0.2in} & rule must return a constraint expression\index{constraint!expression}.
\end{tabular}
\end{center}
The distinction between an objective and constraint expression is simply the
imposition of a lower or upper bound (or both) in the case of a constraint.\index{expression!objective vs. constraint}
The \code{Set} and \code{Param} classes also support direct initialization
with the \code{initialize} argument.\index{set!initialize}\index{parameter!initialize}  This argument allows the user to
specify the data that initializes this class, thereby avoiding the need to
specify a Python function for a \code{rule} argument.\index{data!set}\index{data!parameter}

Example~\ref{ex:concrete5} illustrates the use of \code{rule} and
\code{initialize} arguments in the context of the full spectrum of modeling 
components. This example is more verbose than Example~\ref{ex:concrete4}, and
it does not represent the model in a more flexible manner.  Thus, 
the modeling representation in Example~\ref{ex:concrete4}
might be preferable.  However, Example~\ref{ex:concrete5} avoids the
specification of global data, which promotes encapsulation that may be
important in complex applications.



\section{Abstract Models}
\label{chap:overview-abstractmodels}

\index{abstract model}There are many contexts where a strong separation of model and data is
either helpful or necessary\index{data-model separation}.  Model data may not be immediately available,
or it may be stored in an external database\index{database} or spreadsheet\index{spreadsheet}.  Alternatively, it may simply be
desirable to represent a model in an abstract form that is agnostic to 
the manner in which the data is managed.

A simple strategy for managing models abstractly is to create a Python
function whose arguments reflect the data that is needed to create a
concrete model.  Example~\ref{ex:abstract4} illustrates this approach
using the concrete model developed in Example~\ref{ex:concrete4}. 
The function \code{create\_model} has arguments \code{N}, \code{M},
\code{c}, \code{a} and \code{b} that represent the data needed to create
a concrete model for Formulation~(\ref{eqn:simple-abstract}).  Consequently,
this function provides a strong separation of model and data, and
it can be viewed as a constructor for an abstract model. The code in
both Examples~\ref{ex:concrete4} and \ref{ex:abstract4} can be executed
using the \code{pyomo} command (see Chapter~\ref{chap:overview:pyomo} below),
ultimately yielding identical model instances and therefore 
identical optimization solutions.

Pyomo also supports the definition of \textit{abstract} models that are defined
independent of any specific data.\index{abstract model} Example~\ref{ex:abstract5} illustrates 
the definition of an abstract model in Pyomo for Formulation~(\ref{eqn:simple-abstract}).  
In contrast to concrete models, abstract models strictly define the existence of model
components and the relationships between them (e.g. the fact that set \code{N} is used
to index parameter \code{c})\index{indexed component}. 

The abstract model in Example~\ref{ex:abstract5} differs from the concrete model in 
Example~\ref{ex:concrete5} in three main respects. The first difference is that the class 
\code{AbstractModel} is used instead of \code{ConcreteModel}.\index{AbstractModel component@\code{AbstractModel} component}\index{model!AbstractModel@\code{AbstractModel} component}  This informs Pyomo that this is 
an abstract model that will be constructed with auxiliary data.  Because data is not immediately 
available, the construction of modeling components is deferred until a model instance is generated\index{model!instance}.  The second difference is that 
declarations of data components do not contain references to the data that will be used to 
construct these components.  The final difference is that all objective and constraint 
components must be defined via rules\index{constraint!rule}\index{objective!rule}. This is a requirement in abstract models, since the 
construction of these components ultimately depends on the availability of specific data.

\begin{Xnotebox}
Pyomo does allow for hybrid models where some components are initialized
with data while others are defined abstractly.  Further, Pyomo data
components can be defined with default values that are used when data 
is not specified.  These options are discussed in Chapters~\ref{chap:components2}, \ref{chap:components1}, and \ref{chap:components3}.  However, with few exceptions these hybrid models 
must be defined with the \code{AbstractModel} class, and thus Pyomo treats them 
as an abstract model.
\end{Xnotebox}

Pyomo includes a rich set of options for
initializing an abstract model to create a model instance that can then
be optimized (see Chapter~\ref{chap:data}).\index{model!instance}
A simple strategy is to supply a {\it data command file} that
specifies values for set and parameter data.\index{data command file}
The syntax of Pyomo's data command files is very similar to the data command syntax
supported by AMPL~\cite{AMPL}.\index{algebraic modeling language!AMPL}  A goal of Pyomo is to support the same
syntax for \code{set} and \code{param} data commands that is used in AMPL.
However, the syntax for other commands relating to file inclusion and table
imports and exports is somewhat different (see Chapter~\ref{chap:data}).
Example~\ref{ex:abstract5.dat} shows a file of data commands that be
used to initialize the abstract model in Example~\ref{ex:abstract5}.
This process is described in Section~\ref{chap:overview:pyomo}.


\if 0
% WEH - include this after we implement this functionality.  I think that this will require a new Model base class: ExtensibleModel

\section{Defining Models with Python Classes}

Another strategy for defining models with Pyomo is to create model classes that
are a subclass of \code{Model}.  This approach has the disadvantage that it requires the user to employ more advanced programming features in Python.  However, it has the advantage that the model initialization can be customized
in a problem-specific manner.  In particular, using this approach with Pyomo
eliminates the need to specify rules for constructing modeling components.

Example~\ref{ex:model5.py}
\fi


\section{Optimizing Models}
\label{chap:overview:pyomo}

The previous sections have outlined different strategies for creating
and populating a Pyomo model object\index{model!object}.  After the modeler has specified how this
will be done, he Python commands for doing it are generally invoked by using an executable script.
The most straightforward way is to invoke the \code{pyomo} command that calls a
pre-written script to perform optimization in a standard manner. \index{pyomo command@\code{pyomo} command}
Alternatively, a script can be created to customize the process using solver components
from Pyomo as introduced in Section~\ref{sec:optscripts}.


\subsection{Optimization With The \code{pyomo} Command \label{sec:pyomocommand}}

The Pyomo software distribution includes the \code{pyomo} command that can be used
to construct a Pyomo model, create a model instance from user-supplied data (in
the case of abstract models), apply an optimizer, and summarize the results\index{model!instance}.  For 
example, the following command-line optimizes the \code{concrete1.py} model using
the default LP solver \code{glpk}:\indexsolver{GLPK}
\listing{examples/overview/pyomo.concrete1.sh}{cmd}{4}{4}
Similarly, the following command-line optimizes the \code{abstract5.py} model using
data in \code{abstract5.dat}, also using \code{glpk}:
\listing{examples/overview/pyomo.abstract5.sh}{cmd}{4}{4}
When the \code{pyomo} command loads a user-defined Pyomo model, it by default looks for
the presence of a \code{ConcreteModel} or \code{AbstractModel} named \code{model}
in the supplied Python file. We have used this name for all models introduced in this 
chapter. However, if a name other than \code{model} is used, this can be specified via 
the \code{pyomo} option \code{--model-name=MODEL\_NAME}, \index{model!name} 
e.g., \code{--model-name=mymodel}.

The \code{pyomo} command automatically executes the following steps:
\begin{itemize}
\item Create a model
\item Read the instance data (if applicable)
\item Generate a model instance (if the model is abstract)
\item Apply simple preprocessors to the model instance
\item Apply a solver to the model instance
\item Load the results into the model instance
\item Display the solver results
\end{itemize}
A variety of optional command-line arguments are provided to further guide and 
provide additional information about the optimization process; documentation of
the various available options is available by specifying the \code{--help} option.
Options can control how much or even if debugging information is printed, including 
logging information generated by the optimizer and a summary of the model generated 
by Pyomo. Further, Pyomo can be configured to keep all intermediate files used
during optimization, which facilitates debugging of the model construction process. 
See Chapter~\ref{chap:command} for further details about this command.


\subsection{Optimization Scripts \label{sec:optscripts}}

\index{scripting}
Scripts that control the optimization process provide users with powerful
programmability.  To introduce the topic, we describe a simple Python
script to perform optimization of a Pyomo model instance. Suppose
that the model in Example~\ref{ex:concrete1} is stored in the file
\code{concrete1.py}. The script in Example~\ref{ex:script1} can then be
used to import this model, display the created model, create a solver
interface, perform optimization, and display the results. Note that this
script does not rely on the \code{pyomo.core}\index{pyomo.core package@\code{pyomo.core} package} package directly. Pyomo
is only used to create an optimization model.  Subsequent optimization
and analysis of this model is handled in a generic manner with other
Pyomo packages.

Once a Pyomo model has been created, it can be printed using the \code{pprint} method:
\listing{examples/overview/script1-parts.py}{pprint}{5}{5}
This command summarizes the information in the Pyomo model.  For concrete models, this 
includes the constraint and objective expressions\index{constraint!expression}\index{objective!expression}. In abstract models, this information
is omitted unless the model object has been constructed with externally supplied data.

Before performing optimization, Pyomo needs to perform various preprocessing steps to 
collect variables, simplify expressions, and other pre-optimization tasks.\index{preprocessing}\index{expression!simplification of} Such preprocessing 
is automatically performed by the \code{create} method of both the concrete and abstract Pyomo 
models:
\listing{examples/overview/script1-parts.py}{create}{9}{10}
For concrete models this method simply returns the model instance, with various annotations 
performed by the preprocessor. For abstract models, additional arguments specifying the 
data used to construct the model instance must be supplied. We note that Pyomo's preprocessing capabilities do not 
currently involve the \textit{presolve} operations that are commonly employed by industrial
AMLs to simplify models prior to solution by an optimizer.\index{model!presolve}\index{presolve}

Next, we apply an optimizer to find an optimal solution to our model instance.  For example, 
the GLPK \cite{glpk} linear programming solver can be used within Pyomo as follows:\indexsolver{GLPK}
\listing{examples/overview/script1-parts.py}{opt}{14}{15}
The first line in this example creates a Python object to interface to the GLPK solver.
The Pyomo model instance is then optimized, with the solver returning an object that 
contains the solution(s) generated during optimization. This optimization process is 
executed using components of the \code{pyomo.opt}\index{pyomo.opt package@\code{pyomo.opt} package} package, which manage the setup and execution 
of optimizers.  Additionally, Pyomo optimization plugins are used to manage the
execution of specific optimizers.

Finally, the results of the optimization can be displayed simply by executing the
following command:
\listing{examples/overview/script1-parts.py}{write}{19}{19}
This output is in the YAML format~\cite{YAML}, which is a highly structured data format 
that is also human readable.  
\if 0
Chapter~\ref{chap:results} provides a detailed description 
of the results object and the YAML results format.
\fi

The process for optimizing abstract models (given a data source) is only slightly different.  
Suppose that the model in Example~\ref{ex:abstract5} is stored in the file \linebreak \code{abstract5.py}, 
while the data in Example~\ref{ex:abstract5.dat} is stored in the file\linebreak \code{abstract5.dat}.
The script in Example~\ref{ex:script2} imports this model, creates a model instance from the
supplied data, creates a solver interface, performs optimization, and displays the results.  
The primary difference (other than the use of an abstract model) is the specification of a 
data file, which defines the sets and parameters that are used to construct the model 
instance.

\section{Discussion}

Perhaps the two Python modeling tools most similar to Pyomo are
PuLP~\cite{PuLP} and APLEpy~\cite{APLEpy}.\index{algebraic modeling language!PuLP}\index{algebraic modeling language!APLEpy}  These package both support
the construction of concrete models using Python objects.  However,
Pyomo is clearly distinguished by its ability to support the definition
of abstract models.  Although most of this chapter has focused on
examples using concrete models, most of the development effort in Pyomo
has focused on the mechanisms that support abstract models.  In fact,
the \code{ConcreteModel} class is just a specialization of that mechanism
to immediately construct model components!

Another distinguishing aspect of Pyomo is the fact that models are complete
and self-contained Python objects that a user creates.  We believe that this 
is a feature, since it allows the user to create and managed multiple models 
simultaneously. However, we have heard users complain about the additional 
syntax that this requires. In this way, PuLP and APLEpy may be a bit simpler 
for beginning users, perhaps at the expense of some object-orientation.

In practice, we have observed that users working on abstract models
tend to work with the \code{pyomo} command, while users working on
concrete models tend to work with Python scripts.  The \code{pyomo}
command is relatively mature, since its interface is well-defined.
However, Pyomo's scripting capabilities are still being extended.
Although the simple examples shown in this chapter are straightforward,
more complex examples that we have developed often rely on features of
Pyomo and Pyomo that were not intended for general, public use.  
Consequently, we expect that the scripting capabilities of Pyomo and 
Pyomo will significantly evolve in future releases of this software.


\begin{subappendices}

\section{Examples}

\subsection{Concrete Pyomo Model with Explicit Variables}
\label{ex:concrete1}

A concrete Pyomo model for Formulation~(\ref{eqn:simple-concrete}) using explicit variables.
\listing{examples/samples/pyomo_book/overview_concrete1.py}{}{1}{8}


\subsection{Concrete Pyomo Model with Indexed Variables}
\label{ex:concrete2}

A concrete Pyomo model for Formulation~(\ref{eqn:simple-concrete}) using indexed variables.
\listing{examples/samples/pyomo_book/overview_concrete2.py}{}{1}{7}


\subsection{Concrete Pyomo Model with External Data}
\label{ex:concrete3}

A concrete Pyomo model for Formulation~(\ref{eqn:simple-concrete}) 
with (a) external data declarations and (b) expressions defined using Python's generator syntax.
\listing{examples/samples/pyomo_book/overview_concrete3.py}{}{1}{15}


\subsection{Concrete Pyomo Model with Constraint Rules}
\label{ex:concrete4}

A concrete Pyomo model for Formulation~(\ref{eqn:simple-concrete}) using a constraint rule to create constraints \code{con}.
\listing{examples/samples/pyomo_book/overview_concrete4.py}{}{1}{15}


\subsection{Concrete Pyomo Model with Abstract Component Declarations}
\label{ex:concrete5}

A concrete Pyomo model for Formulation~(\ref{eqn:simple-concrete}) that uses \code{rule} and \code{initialize} arguments for
all modeling components.
\listing{examples/samples/pyomo_book/overview_concrete5.py}{}{1}{25}


\subsection{Using a Function to Construct a Concrete Pyomo Model}
\label{ex:abstract4}

A function that creates a concrete Pyomo model for Formulation~(\ref{eqn:simple-abstract}) using only data provide in the argument list.
\listing{examples/samples/pyomo_book/overview_abstract4.py}{}{1}{17}


\subsection{Abstract Pyomo Model}

\subsubsection{Pyomo Model}
\label{ex:abstract5}

An abstract Pyomo model for Formulation~(\ref{eqn:simple-abstract}).
\listing{examples/samples/pyomo_book/overview_abstract5.py}{}{1}{20}


\subsubsection{Data Commands}
\label{ex:abstract5.dat}

Data commands for the abstract Pyomo model in Example~\ref{ex:abstract5} that are used to generate the concrete model in Formulation~(\ref{eqn:simple-concrete}).
\listing{examples/samples/pyomo_book/overview_abstract5.dat}{}{1}{17}


\subsection{A Python Script that Optimizes a Concrete Pyomo Model}
\label{ex:script1}

A Python script that creates the concrete Pyomo model in Example~\ref{ex:concrete1} and performs optimization using the GLPK linear programming solver.
\listing{examples/samples/pyomo_book/overview_script1.py}{}{1}{12}


\subsection{A Python Script that Optimizes an Abstract Pyomo Model}
\label{ex:script2}

A Python script that creates a model instance from the abstract Pyomo model in Example~\ref{ex:abstract5} and performs optimization using the GLPK linear programming solver.
\listing{examples/samples/pyomo_book/overview_script2.py}{}{1}{12}


\end{subappendices}
